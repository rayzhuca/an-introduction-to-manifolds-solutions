\section{Euclidean Spaces}

\subsection{Smooth Functions on a Euclidean Space}

\begin{exercise}{1.2}
Let $f(x)$ be the function on R defined in Example 1.3.
\begin{enumerate}[label=(\alph*)]
    \item Show by induction that for $x > 0$ and $k \geq 0$, the kth derivative $f^{(k)}(x)$ is of the form
    $p_{2k}(1/x)e^{-1/x}$ for some polynomial $p_{2k}(y)$ of degree $2k$ in $y$.
    \item Prove that $f$ is $C^\infty$ on $\R$ and that $f^{(k)}(0)=0$ for all $k \geq 0$.
\end{enumerate}
\end{exercise}

\begin{enumerate}[label=(\alph*)]

    \item Suppose $x > 0$. We induct on $k \geq 0$. For the base case $k = 0$, $f^{(0)}(x) = e^{-1/x} = p_{0}(1/x) e^{-1/x}$ where 
    $p_0 (1/x) = 1$. Now assume the inductive hypothesis for $k$. Notice that 
    \begin{align}
        f^{(k+1)}(x) &= \dv{x}( p_{2k}\left(\frac{1}{x}\right) e^{-1/x} ) \\ 
        &=  p_{2k}\left( \frac{1}{x} \right) \cdot e^{-1/x} \frac{1}{x^2} - p_{2k}'\left(\frac{1}{x}\right) \frac{1}{x^2} \cdot e^{-1/x} \\
        &= \left( p_{2k}(1/x) \frac{1}{x^2} - p_{2k}'\left(1/x\right) \frac{1}{x^2} \right) e^{-1/x} .
    \end{align}

    Let $p_{2(k+1)} = p_{2k} (x) x^2 - p_{2k}'(x) x^2$ so that $f^{(k+1)}(x) = p_{2(k+1)}(1/x) e^{-1/x}$. 
    Clearly, $p_{2(k+1)}$ is of degree $2(k+1)$ since $p_{2k}$ is of 
    degree $2k$ and $p_{2k}'$ is at most of degree $2k - 1$.
    Therefore, the claim is true for $k \geq 0$.

    \item We first establish that $f^{(k)}(0) = 0$ via induction.
    
    The base case $k=0$ is trivial. Assume our desired result is true for $k$ now.
    To prove $f^{(k+1)}(0) = 0$, we first show $\lim_{x \to 0^+} f^{(k)}(x) / x = 0$. But 
    $\lim_{x \to 0^+} f^{(k)} (x) = \lim_{x \to \infty} f^{(k)} (1/x)$. Combined with part (a), 
    \begin{equation} \label{eq:1-2-a}
        \lim_{x \to 0^+} f^{(k)} (x) / x = \lim_{x \to \infty} x f^{(k)} (1/x) = \lim_{x \to \infty} xp_{2k}(x) e^{-x} = \lim_{x \to \infty} \frac{xp_{2k}(x)}{e^x}.
    \end{equation}
    The last expression in Equation~\ref{eq:1-2-a} is $0$ because exponentials grow faster than any polynomial. (You can show 
    this by repeatedly applying L'Hospital's Rule.) 

    Obviously, $\lim_{x \to 0^-} f^{(k)}(x) / x = 0 / x = 0$. Since the left and right derivative agrees and $f^{(k)}$ 
    is continuous at $0$ because it is differentiable at $0$, $f^{(k+1)} = 0$.

    Now we have shown $f^{(k)}(0) = 0$ for all $k \geq 0$, we can conclude that $f \in C^\infty(\R)$ 
    because $f$ is clearly $C^\infty$ on $(-\infty, 0)$, on $0$ by our previous result, and on $(0, \infty)$ by part (a).

\end{enumerate}